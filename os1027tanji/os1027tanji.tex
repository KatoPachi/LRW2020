\documentclass[../root]{subfiles}
\graphicspath{{_images/}{../_images/}}

\begin{document}

    \chapter{Shocking Racial Attitudes: Black G.I.s in Europe}

    \begin{shortsummary}
        \begin{itemize}
            \item \authoryear{Schindleretal2020}
            \item \RQ{Can attitudes toward minorities be changed, or a constant of the human condition?}
            \item \answer{Regression using empirical dataset of US army units in the UK.}
            \item \result{Presence of African American soldiers in the UK during World War II reduced anti-minority prejudice, as a result of the positive interactions.}
        \end{itemize}
    \end{shortsummary}

    \section{Introduction}

    \paragraph{Prejudicial attitudes toward minority groups}

    \begin{itemize}
      \item Prejudicial attitudes toward minority groups are widespread, and persists over the very long run (Voigtl\"{a}nder and Voth, 2012; Acharya et al., 2016)
      \item Less is known about:
      \begin{itemize}
        \item what it takes to change such attitudes.
        \item whether any such changes in attitudes might themselves persist.
      \end{itemize}
      \item This is important to understand the consequences
      \begin{itemize}
        \item social conflict
        \item hate crime
        \item labour and goods market discrimination.
      \end{itemize}
    \end{itemize}

    \paragraph{G.I.s in the UK}

    \begin{itemize}
      \item During World War II, the UK played host to over one and a half million US troops, including 150,000 African Americans.
      \begin{itemize}
        \item Serving with non-combat support duties such as transport and supply.
        \item Many Britons saw and interacted with non-whites for the very first time.
        \begin{itemize}
          \item Frequently found in pubs, dance halls, and Restaurants (Millgate, 2010)
        \end{itemize}
      \end{itemize}
      \item A newly constructed dataset of US military bases in the UK and present-day measures of anti-minority preferences show:
      \begin{itemize}
        \item Individuals in areas of the UK where black troops were posted are more tolerant towards minorities 60 years ofter the last troops left.
        \begin{enumerate}
          \item Fewer members of the British National Party (BNP).
          \item Residents are less likely to vote Conservative in local elections.
          \item Less implicit anti-black bias measured by Implicit Association Test (IAT) scores.
        \end{enumerate}
      \end{itemize}
    \end{itemize}

    \paragraph{Contribution}

    \begin{itemize}
      \item Contact Hypothesis (Allport, 1954): Contact with minorities can reduce prejudice.
      \begin{itemize}
        \item Boisjoly et al.(2006); Carell et al.(2019); Burns et ak.(2016): assigning non-white roommates to white students at higher education has positive effects on the attitudes toward non-whites.

        $\Rightarrow$ meaningful effect on broader cross-section of the population.
      \end{itemize}
      \item Cultural norms: Preferences are endogenous to social and family environments.
      \begin{itemize}
        \item Path of preference formation: vertical and horizontal transmission of values (Bisin and Verdier, 2001).
        \item Very long-run persistence in preferences (Voigtl\"{a}nder and Voth, 2012; Archarya, 2016)
        \item This paper suggests contact as an important potential source for persistent changes in attitudes/cultural norms (Fouka, 2020).
      \end{itemize}
      \item Historical determinants of support for far-right parties
      \begin{itemize}
        \item Political alienation increase support for radical far-right parties.(Vlanchos, 2017)
        \item The effect is persistent (Ochsner and Roesel, 2020; Cantoni et al., 2019)
        \item This paper suggests a high level intergenerational correlation in attitudes towards migration and support for far-right parties (Avdeenko and Siedler, 2017).
      \end{itemize}
      \item Effect of a historical event on implicit attitudes as measured by a computerized IAT
      \begin{itemize}
        \item Implicit attitudes against minority groups are increasingly used in the economics literature to measure bias, and predictive of behavior (Greenwald and Banaji, 1995; Lowes et al., 2015). 
        \item Historic events can shape implicit attitudes (Lowes and Montero, 2016; Lowes et al., 2017).
      \end{itemize}
    \end{itemize}

    \section{Historical Overview}

    \section{Evidence on Contact}

    \section{Estimation Framework}

    \section{Long-Term Effects on BNP Menbership}

    \section{Additional Outcome Measures}

    \section{Mechanism}

    \section{Conclusion}


    %\begin{figure}[ht]
    %    \centering
    %    \includegraphics[scale = 1]{os1027tanji/}
    %    \caption{}
    %    \label{}
    %\end{figure}

    \biblio

\end{document}
